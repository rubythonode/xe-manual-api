\subsection*{Issue 작성}

Issue 작성 시 참고해주세요.


\begin{DoxyItemize}
\item 작성하려는 이슈가 이미 있는지 검색 후 등록해주세요. 비슷한 이슈가 있다면 댓글로 추가 내용을 덧붙일 수 있습니다.
\item 이슈에는 하나의 문제 또는 제안을 작성해주세요. 절대 하나의 이슈에 2개 이상의 내용을 적지마세요.
\item 이슈는 가능한 상세하고 간결하게 작성해주세요.
\begin{DoxyItemize}
\item 필요하다면 화면을 캡처하여 이미지를 업로드할 수 있습니다.
\end{DoxyItemize}
\end{DoxyItemize}

\subsection*{Pull request(\+P\+R)}


\begin{DoxyItemize}
\item {\ttfamily master} 브랜치의 코드는 수정하지마세요.
\item P\+R은 {\ttfamily develop} 브랜치만 허용합니다.
\item {\ttfamily develop} 브랜치를 부모로 한 토픽 브랜치를 활용하면 편리합니다.
\end{DoxyItemize}

\subsection*{Coding Guidelines}

코드를 기여할 때 Coding conventions을 따라야합니다.


\begin{DoxyItemize}
\item 모든 text 파일의 charset은 B\+O\+M이 없는 U\+T\+F-\/8입니다.
\item newline은 U\+N\+IX type을 사용합니다. 일부 파일이 다른 type을 사용하더라도 절대 고치지 마세요!
\item 들여쓰기는 1개의 탭으로 합니다.
\item class 선언과 function, if, foreach, for, while 등 중괄호의 {\ttfamily \{\}}는 다음 줄에 있어야 합니다.
\begin{DoxyItemize}
\item 마찬가지로 선언 다음에는 공백을 두지 않습니다. ex) C\+O\+R\+R\+E\+CT {\ttfamily if(...)}, I\+N\+C\+O\+R\+R\+E\+CT {\ttfamily if (...)}
\end{DoxyItemize}
\item {\bfseries Coding convention에 맞지 않는 코드를 발견 하더라도 목적과 관계 없는 코드는 절대 고치지 마세요.} 
\end{DoxyItemize}