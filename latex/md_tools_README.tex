\subsection*{Build}

X\+E패키징을 위해 grunt.\+js를 이용한 Task로 작성되어 있으며 $\ast$nix 환경에서 Node를 구동할 수 있어야 한다.

\subsubsection*{의존 패키지 설치 (Mac)}

\subsubsection*{1. brew}

\href{http://brew.sh}{\tt http\+://brew.\+sh} 페이지의 'Install Homebrew'를 참고하여 설치

\subsubsection*{2. node.\+js}

\begin{quote}
brew install node \end{quote}


\subsubsection*{3. grunt-\/cli}

\begin{quote}
npm install -\/g grunt-\/cli \end{quote}


\subsection*{node 모듈 설치}

grunt task 수행에 필요한 node 모듈의 설치를 먼저 수행한다.

X\+E 패키지의 루트… index.\+php 또는 Gruntfile.\+js 파일이 있는 곳에서 다음 명령으로 의존 모듈이 자동 설치된다.

\begin{quote}
npm install \end{quote}


\subsection*{Build}

build 수행 전 Minified 파일이 갱신되도록 패키징 하려는 브랜치에서 아래 Task 중 {\ttfamily grunt minify}를 반드시 먼저 수행하도록 한다.

build 명령으로 zip, tgz 포맷으로 패키징을 수행하며 지정한 대상의 변경된 파일만을 묶은 changed 파일을 함께 생성한다.

\begin{quote}
grunt build{\ttfamily \+:from}\+:{\ttfamily to} \end{quote}


{\ttfamily from}, {\ttfamily to}에는 commit hash 또는 tag, branch를 지정할 수 있다. 패키지는 {\ttfamily to}에 지정한 대상을 기준으로하며 {\ttfamily from}과 {\ttfamily to}사이에 변경된 파일들 changed 파일을 별도로 생성한다.

\begin{quote}
grunt build\+:{\ttfamily old\+\_\+tag}\+:{\ttfamily current\+\_\+tag} \end{quote}


이와 같이 지정하면 {\ttfamily old\+\_\+tag}로부터 {\ttfamily current\+\_\+tag} 사이의 변경된 파일만을 묶은 {\ttfamily xe.\+current\+\_\+tag.\+changed.$\ast$} 파일과 {\ttfamily xe.\+current\+\_\+tag.$\ast$}파일을 생성한다.

{\ttfamily from}을 생략하여 {\ttfamily build\+:master}(master는 branch이다)와 같이 지정하면 {\ttfamily master}의 최신 상태로 빌드하며 changed 파일을 생성하지 않는다.

\subsubsection*{Build 수행 시 포함하는 패키지}

Build 수행 시 일부 확장 기능을 가져와 함게 패키징한다. 지정한 각 저장소의 master 브랜치로부터 코드를 가져오므로 {\bfseries Build 수행 전에 각 저장소의 master 상태를 확인하도록 한다.}


\begin{DoxyItemize}
\item board 모듈
\begin{DoxyItemize}
\item \href{https://github.com/xpressengine/xe-module-board}{\tt https\+://github.\+com/xpressengine/xe-\/module-\/board}
\end{DoxyItemize}
\item krzip 모듈
\begin{DoxyItemize}
\item \href{https://github.com/xpressengine/xe-module-krzip}{\tt https\+://github.\+com/xpressengine/xe-\/module-\/krzip}
\end{DoxyItemize}
\item syndication 모듈
\begin{DoxyItemize}
\item \href{https://github.com/xpressengine/xe-module-syndication}{\tt https\+://github.\+com/xpressengine/xe-\/module-\/syndication}
\end{DoxyItemize}
\end{DoxyItemize}

\subsection*{Task}

.js, .css, .php 파일들에 대해 문법 검사 및 권장 코드를 확인할 수 있으며 minify 등의 작업을 수행할 수 있다.

\subsubsection*{Lint}

.js, .css, .php 파일에 대해 문법 검사 등을 수행한다. \begin{quote}
grunt lint \end{quote}


다음과 같이 선택적으로 수행할 수 있다.

\paragraph*{J\+S Lint (jshint)}

{\ttfamily lint}가 아닌 {\ttfamily hint}임에 주의. \begin{quote}
grunt jshint \end{quote}


\paragraph*{C\+S\+S Lint}

\begin{quote}
grunt csslint \end{quote}


\subsubsection*{Minify}

.js, .css 파일의 공백을 지우는 등 minify 동작을 수행할 수 있으며 대상은 Gruntfile.\+js 파일에 정의되어 있다.

\begin{quote}
grunt minify\end{quote}
